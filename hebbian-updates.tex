% A S Sharma 2016

\documentclass[onecolumn, twoside, 11pt]{article}

\usepackage{parskip}
\usepackage{amsmath,amsfonts,amssymb}
\usepackage{graphicx}
\usepackage{array}
\usepackage{dcolumn}

\usepackage{fontspec}
\usepackage{xunicode}
\usepackage{xltxtra}
\defaultfontfeatures{Mapping=tex-text}
%\setmainfont[Ligatures={Common,TeX}]{Bembo Std}
%\setsansfont[Ligatures={Common,TeX}]{Helvetica Neue LT Pro}
\usepackage[final]{microtype}

%--------------------------------------------------------------------

\newcommand{\norm}[1]{\lVert#1\rVert}
\newcommand{\vbar}{\Big\vert}
\newcommand{\inprod}[2]{\left<{#1},{#2}\right>}

\newcommand\va{\mathbf{a}}
\newcommand\vb{\mathbf{b}}
\newcommand\vc{\mathbf{c}}
\newcommand\vd{\mathbf{d}}
\newcommand\ve{\mathbf{e}}
\newcommand\vf{\mathbf{f}}
\newcommand\vg{\mathbf{g}}
\newcommand\vh{\mathbf{h}}

\newcommand\vk{\mathbf{k}}

\newcommand\vs{\mathbf{s}}

\newcommand\vu{\mathbf{u}}
\newcommand\vv{\mathbf{v}}
\newcommand\vw{\mathbf{w}}

\newcommand\vx{\mathbf{x}}
\newcommand\vy{\mathbf{y}}
\newcommand\vz{\mathbf{z}}

\newcommand\vhg{\hat{\mathbf{g}}}
\newcommand\vhk{\hat{\mathbf{k}}}
\newcommand\vhu{\hat{\mathbf{u}}}
\newcommand\vhx{\hat{\mathbf{x}}}
\newcommand\vhy{\hat{\mathbf{y}}}
\newcommand\vhz{\hat{\mathbf{z}}}

\newcommand\vtg{\tilde{\mathbf{g}}}
\newcommand\vtu{\tilde{\mathbf{u}}}
\newcommand\vtx{\tilde{\mathbf{x}}}
\newcommand\vty{\tilde{\mathbf{y}}}
\newcommand\vtz{\tilde{\mathbf{z}}}

\newcommand\mA{\mathbf{A}}
\newcommand\mB{\mathbf{B}}
\newcommand\mL{\mathbf{L}}
\newcommand\mU{\mathbf{U}}
\newcommand\mF{\mathbf{F}}

\newcommand\Z{\mathbb{Z}}
\newcommand\N{\mathbb{N}}
\newcommand\R{\mathbb{R}}
\newcommand\C{\mathbb{C}}

%--------------------------------------------------------------------

\title{Notes on resolvent modes from simulation data using Hebbian updates}
\author{A S Sharma}

\begin{document}
\maketitle

\section{Singular value decomposition using Hebbian updates}

Based on~\cite{conf/eacl/Gorrell06}, the following algorithm will generate the singular vectors of a matrix or linear operator $R: \mF \rightarrow \mU$.
The notation in that paper is a bit confusing so I will try to summarise and clarify it here.

Let $u \in \mU$ and $f \in \mF$ be a matching data vector pair satisfying $u = Rf$. The inner product in $\mU$ is denoted $\inprod{x}{y}_\mU$ and similarly for $\mF$.

Let $c^{u*}_i$ be the true $i$th left singular vector of $R$ to be found and $c^{f*}_i$ be the corresponding true right singular vector.
Let $c^u_i$ be the current approximation to $c^{u*}_i$ and define $c^f_i$ similarly.
The notation $\Delta c^u_i$ indicates an update (change) to the approximation $c^u_i$.

According to \cite{conf/eacl/Gorrell06} (eqs 19-20), updates of $c^u$ and $c^f$ may be performed using
\begin{eqnarray}
    \label{eq:updates}
    \Delta c^u_i = \inprod{c^f_i}{f}_\mF ( u - \sum_{j<i} \inprod{u}{c^u_j}_\mU c^u_j), \\
    \Delta c^f_i = \inprod{c^u_i}{u}_\mU ( f - \sum_{j<i} \inprod{f}{c^f_j}_\mF c^f_j).
\end{eqnarray}

My understanding is that the approximations $c^u_i$ and $c^f_i$ are proven to converge to the true singular vectors of $R$ given `enough' iterations and data vectors $(u,f)$ that span $(\mU,\mF)$.

Note that we do not need direct access to $R$ --- only enough data vector pairs. This opens up the possibility of using random vectors, or even direct numerical simulation datasets, with low storage and computational requirements.

\section{Application to linearised code}

In the case that we have access to a linearised code we would still need to form $R$ in order to generate test vector pairs. This is a slight improvement on the loop in the algorithm of~\cite{doi:10.2514/6.2022-1329}.

\section{Application to nonlinear turbulent simulations}

What really got my attention, however, is the possibility of using this algorithm on-line with data generated from a turbulent simulation.

To see why this is possible, recall that the resolvent formulation used in \cite{McKeon.Sharma:2010} of the Navier-Stokes equations has to parts.
The linear part is (in the frequency-domain)
\begin{equation}
    \label{eq:resolvent}
    u(\omega) = R(\omega)f(\omega)
\end{equation}
and the nonlinear part is
\begin{equation}
    \label{eq:nonlinear}
    f(t) = u(t) \cdot \nabla u(t).
\end{equation}

Both must be true simultaneously and always.

Since the updates require only the pairs $(u,f)$, instead of generating $u$ from $f$ via the resolvent using \eqref{eq:resolvent}, we can generate $f$ from $u$ via their nonlinear relationship \eqref{eq:nonlinear}. The frequency-domain pairs $(u(\omega), f(\omega))$ can then be generated from snapshot matrices of $u(t)$ and $f(t)$.

To do this, first form the snapshot matrix $U_1$,
\[U_1 = \left[
        \begin{array}{ccccc}
            u(t_1) & u(t_2) & u(t_3) & \ldots & u(t_N)
        \end{array}
    \right].
\]
Its discrete Fourier transform is the matrix of frequencies which form a set (across frequencies) of snapshots in $\mU$.
\[ F_1 = DFT(U_1) = \left[
        \begin{array}{ccccc}
            u(\omega_1) & u(\omega_2) & u(\omega_3) & \ldots 
        \end{array}
    \right].
\]
The frequencies that can be resolved are determined by the timestep and $N$.
Similarly, form the corresponding snapshot matrix $F_1$, with $DFT(F_1)$ giving a right test vector per frequency in $\mF$.

Inspired by Welch's method, we can generate the next snapshot matrix by shifting time by one (or more) steps,
\[U_2 = \left[
        \begin{array}{ccccc}
            u(t_2) & u(t_3) & u(t_4) & \ldots & u(t_{N+1})
        \end{array}
    \right].
\]
The next set of test vector pairs (one per frequency) is given by the DFT of $U_2$ and $F_2$, and so on.

This process gives one test vector pair per frequency, per snapshot matrix and can be fed to \eqref{eq:updates} until convergence.

\section{Other approaches}

I haven't read \cite{towne2015proceedings} but it and following work may be related.

\bibliography{hebbian-updates.bib}
\bibliographystyle{ieeetr}
\end{document}
